\section{Remaining flaws}

In this section other flaws that I detected are described along with some ideas
to fix them and to improve the code overall.

\subsection*{SQL driver}
The SQL driver is hard-coded to the HyperSQL JDBC driver, what if we want to use
another type of SQL database (e.g. MySQL)? We could add two properties in
\emph{web\_portal.cfg} which would then be used in the \emph{getConnection()}
method of \emph{SQLDatabase} to retrieve the JDBC driver and the subprotocol
(e.g. in \emph{jdbc:hsqldb:...}, \emph{hsqldb} is the subprotocol).

\subsection*{Data \& database coupling}
Data is coupled with the place it will be stored in. It can be seen in methods
such as \emph{asSql()}, \emph{asSqlDelete()}, \emph{asXML()}, or the
constructors with a \emph{ResultSet} parameter.\\

With the use of the \emph{gson} library, we do not have such a problem for the
JSON format. This library handles transformation from JSON to object and
vice-versa by using reflection.\\

As of now, XML documents and parts as well as SQL queries are produced by
concatenating Strings (e.g.Data, Pages). We could instead use an XML builder
(e.g. dom4j \cite{cite:dom4j}) and an SQL builder. First, the code would be
easier to read, and secondly, it would be useful to reuse XML and SQL parts.\\

To help reusing SQL parts, they could be moved into repositories (Repository
pattern, can be seen for example in the Doctrine ORM \cite{cite:doctrine}).
It would help move the data itself from the way we retrieve it.

\subsection*{Presentation layer \& application layer}

The presentation layer possesses multiple Pages. Each Page have \emph{doPost()}
and \emph{doGet()} methods to handle requests for these pages. This layer should
just be concerned about filling and sending the correct UI, but it is not the
case in this application.\\

The presentation tier calls methods from the logic tier to retrieve data from
the database. Then, the data is manipulated (e.g. validation, instantiating the
correct data) and finally inserted in the UI.\\

\begin{lstlisting}[caption={data validation in AdministrationPage}]
	public String doPost(	HttpServletRequest	request,
				HttpServletResponse	response) {
		...
				String validate = request
					.getParameter("Validate");
				if (validate == null)
					appFacade.updateRawData(rawData);
				else {
					appFacade.deleteRawData(rawData);
					appFacade.add(rd);
				}
        ...
    }
\end{lstlisting}
\

There are multiple problems here, first, the application layer has a limited
purpose here and is just used as a wrapper for methods of the database layer. It
does not compute anything useful from the data to give to the presentation layer.
The second problem which is a result of the first one is that the presentation
layer gets the data and compute things with it before sending the UI.\\

To summarise, the application layer is mostly useless, the presentation layer
does more than it should and is more or less calling directly methods from the
database layer (through methods from \emph{ApplicationFacade} which are just
wrappers). So, the presentation layer should delegate the manipulation of data
to \emph{ApplicationFacade}.

\newpage
