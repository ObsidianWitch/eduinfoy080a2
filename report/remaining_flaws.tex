\section{Remaining flaws}

In this section other flaws that I detected are described along with some ideas
to fix them and to improve the code overall.

\subsection*{SQL driver}
The SQL driver is hard-coded to the HyperSQL JDBC driver, what if we want to use
another type of SQL database (e.g. MySQL)? We could add two properties in
\emph{web\_portal.cfg} which would then be used in the \emph{getConnection()}
method of \emph{SQLDatabase} to retrieve the JDBC driver and the subprotocol
(e.g. in \emph{jdbc:hsqldb:...}, \emph{hsqldb} is the subprotocol).

\subsection*{Data \& database coupling}
Data is coupled with the place it will be stored in. It can be seen in methods
such as \emph{asSql()}, \emph{asSqlDelete()}, \emph{asXML()}, or the
constructors with a \emph{ResultSet} parameter.\\

With the use of the \emph{gson} library, we do not have such a problem for the
JSON format. This library handles transformation from JSON to object and
vice-versa by using relection.\\

As of now, XML documents and parts as well as SQL queries are produced by
concatenating Strings (e.g.Data, Pages). We could instead use an XML builder
(e.g. dom4j \cite{cite:dom4j}) and an SQL builder. First, the code would be
easier to read, and secondly, it would be useful to reuse XML and SQL parts.\\

To help reusing SQL parts, they could be moved into repositories (Repository
pattern, can be seen for example in the Doctrine ORM \cite{cite:doctrine}).
It would help move the data itself from the way we retrieve it.

% TODO
%\begin{framehint}
%On a side note, XML pages are created with the help of XSLT which transforms the
%given XML input into an HTML page by using an XSL document
%(\emph{web\_portal.xsl}).
%\end{framehint}

\newpage
