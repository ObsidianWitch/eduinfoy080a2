\subsection{UserProfile \& subclasses}

\emph{ExpensiveSubscription}, \emph{CheapSubscription},
\emph{FreeSubscription}, \emph{ExpertAdministator},
\emph{ExternalAdministrator}, \emph{RegularAdministrator} and
\emph{Operator} contain a lot of code duplication (constructors,
\emph{asSql()}, \emph{asSqlUpdate()} and \emph{asXml()}), only a few strings
are different.\\

First, the \emph{getType()} abstract method was added to the \emph{UserProfile}.
All its subclasses need to implement it by indicating the type of user.
It will be used further below as an uniform way to know where the data must be
added.\\

\begin{lstlisting}
public class UserProfile extends Data {
    ...
    protected abstract String getType();
    ...
}

public class CheapSubscription extends RegularUser {
    ...
    
    @Override
	protected String getType() {
		return "CheapSubscription";
	}
    ...
}
\end{lstlisting}
\

With the \emph{getType()} method we can modify and move the \emph{asSql()},
\emph{asSqlUpdate()}, and \emph{asXml()} methods from the subclasses to
\emph{UserProfile}. The constructors can also be moved.\\

\begin{lstlisting}
public class UserProfile extends Data {
	public UserProfile(HttpServletRequest request) { ... }
    public UserProfile(ResultSet rs) throws SQLException, ParseException {
        ...
    }
    public UserProfile(String username, String password, String firstName,
		String lastName, String emailAddress, Date lastLogin)
    { ... }
        
    public String asXml() {
		return	"<" + getType() + ">" +
			"<username>" + normalizeXml(username) + "</username>" +
			// password is not returned,
			// as it should only be used internally
			"<firstName>" +
			normalizeXml(firstName) +
			"</firstName>" +
			"<lastName>" + normalizeXml(lastName) + "</lastName>" +
			"<emailAddress>" +
			normalizeXml(emailAddress) +
			"</emailAddress>" +
			"<lastLogin>" + df.format(lastLogin) + "</lastLogin>" +
			"</" + getType() + ">";
	}

	/**
	 * Returns an SQL INSERT string that allows the system to add
	 * the account to a relational database.
	 */
	public String asSql() {
		return	"INSERT INTO " + getType() + " (Username, " +
			"Password, FirstName, LastName, EmailAddress, " +
			"LastLogin) VALUES (\'" + normalizeSql(username) +
			"\', \'" + normalizeSql(password) +"\', \'" +
			normalizeSql(firstName) + "\', \'" +
			normalizeSql(lastName) + "\', \'" +
			normalizeSql(emailAddress) + "\', \'" +
			df.format(lastLogin) + "\');";
	}

	/**
	 * Returns an SQL UPDATE string that allows the system to update
	 * the account in a relational database.
	 */
	public String asSqlUpdate() {
		return  "UPDATE " + getType() + " SET Password = \'" +
			normalizeSql(password) + "\', FirstName = \'" +
			normalizeSql(firstName) + "\', LastName = \'" +
			normalizeSql(lastName) + "\', EmailAddress = \'" +
			normalizeSql(emailAddress) + "\', LastLogin = \'" +
			df.format(lastLogin) + "\' " + "WHERE Username = \'" +
			normalizeSql(username) + "\';";
	}
}
\end{lstlisting}

\subsubsection{Type refactoring}

Further code refactoring was done, a \emph{String} field containing
the type of user was added to the \emph{UserProfile} class.
\emph{UserProfile} was also made instantiable (previously was Abstract).
Then, the following classes were removed: \emph{ExpensiveSubscription},
\emph{CheapSubscription}, \emph{FreeSubscription},
\emph{ExpertAdministator}, \emph{ExternalAdministrator},
\emph{RegularAdministrator} and \emph{Operator} classes. By doing so, it
is easier to add new types of users.\\

But these modifications are not sufficient, first of all the
\emph{UserProfile} subclasses implemented the \emph{getDefaultPage()}
method. To still be able to retrieve the deault page, an enum was
created. The \emph{UserTypes} enum contains the different existing types
(e.g. ExpertAdministrator), with each type containing its default page.\\

% TODO caption: enum & default page
\begin{lstlisting}
public class UserProfile extends Data {
    public enum UserTypes {
        ExpertAdministrator("web_portal?Page=Administration"),
        ExternalAdministrator("web_portal?Page=Administration"),
        RegularAdministrator("web_portal?Page=Administration"),
        CheapSubscription("web_portal?Page=Query"),
        ExpensiveSubscription("web_portal?Page=Query"),
        FreeSubscription("web_portal?Page=Query"),
        Operator("web_portal?Page=Operation");
        
        private String defaultPage;
 
        private UserTypes(String defaultPage) {
            this.defaultPage = defaultPage;
        }
    }
    
    ...
    
    public String getDefaultPage() {
        return UserTypes.valueOf(type).defaultPage;
    }
    
    ...
}
\end{lstlisting}
\

By doing the refactoring described above, we had to modify multiple classes
impacted by the changes.\\

The SQL database now contains a \emph{UserProfile} table, and all the tables
related to the subclasses were removed. It was easy to do, since all the
deleted tables contained the same fields. Consequently, the \emph{asSql()}
and \emph{asSqlUpdate()} methods were also modified in \emph{UserProfile} as
seen below.\\

% TODO caption: UserProfile asSql() diff
\begin{lstlisting}[language=diff]
public String asSql() {
-   return  "INSERT INTO " + type + " (Username, " +
-       "Password, FirstName, LastName, EmailAddress, " +
-       "LastLogin) VALUES (\'" + normalizeSql(username) +
-       "\', \'" + normalizeSql(password) +"\', \'" +
-       normalizeSql(firstName) + "\', \'" +
-       normalizeSql(lastName) + "\', \'" +
-       normalizeSql(emailAddress) + "\', \'" +
-       df.format(lastLogin) + "\');";
+   return  "INSERT INTO userProfile (Username, " +
+       "Password, FirstName, LastName, EmailAddress, " +
+       "LastLogin, Type) VALUES (\'" + normalizeSql(username) +
+       "\', \'" + normalizeSql(password) +"\', \'" +
+       normalizeSql(firstName) + "\', \'" +
+       normalizeSql(lastName) + "\', \'" +
+       normalizeSql(emailAddress) + "\', \'" +
+       df.format(lastLogin) + "\', \'" +
+       normalizeSql(type) + "\');";
}
\end{lstlisting}
\

The \emph{findUser()} method in \emph{UserSQLDatabase} do not need to
query all the user tables anymore, since they were merged into the
unique \emph{UserProfile} table (same for \emph{userExists()}).\\

\begin{lstlisting}
public UserProfile findUser(String username) throws DatabaseException {
    ...
        ResultSet rs = statement.executeQuery(
            "SELECT * FROM UserProfile WHERE " +
            "Username = \'" + username + "\';"
        );
        
        if (rs.first()) {
            return new UserProfile(rs);
        }
    ...
}
\end{lstlisting}
\

As for the UI, in the \emph{RegistrationPage} class we do not need to
check which subscription type was selected anymore, the request's result
contains the type of subscription and is passed to \emph{UserProfile}'s
constructor.\\

% TODO caption: diff RegistrationPage
\begin{lstlisting}[language=diff]
-           UserProfile up;
-           switch (request
-               .getParameter("Subscription")
-               .charAt(0)) {
-
-               case 'F':
-                   up = new FreeSubscription(request);
-                   break;
-               case 'C':
-                   up = new CheapSubscription(request);
-                   break;
-               case 'E':
-                   up = new ExpensiveSubscription(request);
-                   break;
-               default:
-                   throw new ApplicationException(
-                       "You did not provide a valid " +
-                       "subscription type.  Please " +
-                       "try again.");
-           }
+           UserProfile up = new UserProfile(request);
\end{lstlisting}

% TODO caption: diff web_portal.xsl
\begin{lstlisting}[language=diff]
- <input type="radio" name="Subscription" value="Free" checked="checked" />Free Subscription<br />
- <input type="radio" name="Subscription" value="Cheap" />Cheap Subscription<br />
- <input type="radio" name="Subscription" value="Expensive" />Expensive Subscription<br />
+ <input type="radio" name="Type" value="FreeSubscription" checked="checked" />Free Subscription<br />
+ <input type="radio" name="Type" value="CheapSubscription" />Cheap Subscription<br />
+ <input type="radio" name="Type" value="ExpensiveSubscription" />Expensive Subscription<br />
\end{lstlisting}

\newpage
