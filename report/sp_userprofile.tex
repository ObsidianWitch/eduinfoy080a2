\subsection{UserProfile \& subclasses}

\emph{ExpensiveSubscription}, \emph{CheapSubscription},
\emph{FreeSubscription}, \emph{ExpertAdministator},
\emph{ExternalAdministrator}, \emph{RegularAdministrator} and
\emph{Operator} contain a lot of code duplication (constructors,
\emph{asSql()}, \emph{asSqlUpdate()} and \emph{asXml()}), only a few strings
are different.\\

First, the \emph{getType()} abstract method was added to the \emph{UserProfile}.
All its subclasses need to implement it by indicating the type of user.
It will be used further below as an uniform way to know where the data must be
added.

\begin{lstlisting}
public class UserProfile extends Data {
    ...
    protected abstract String getType();
    ...
}

public class CheapSubscription extends RegularUser {
    ...
    
    @Override
	protected String getType() {
		return "CheapSubscription";
	}
    ...
}
\end{lstlisting}
\

With the \emph{getType()} method we can modify and move the \emph{asSql()},
\emph{asSqlUpdate()}, and \emph{asXml()} methods from the subclasses to
\emph{UserProfile}. The constructors can also be moved.

\begin{lstlisting}
public class UserProfile extends Data {
	public UserProfile(HttpServletRequest request) { ... }
    public UserProfile(ResultSet rs) throws SQLException, ParseException {
        ...
    }
    public UserProfile(String username, String password, String firstName,
		String lastName, String emailAddress, Date lastLogin)
    { ... }
        
    public String asXml() {
		return	"<" + getType() + ">" +
			"<username>" + normalizeXml(username) + "</username>" +
			// password is not returned,
			// as it should only be used internally
			"<firstName>" +
			normalizeXml(firstName) +
			"</firstName>" +
			"<lastName>" + normalizeXml(lastName) + "</lastName>" +
			"<emailAddress>" +
			normalizeXml(emailAddress) +
			"</emailAddress>" +
			"<lastLogin>" + df.format(lastLogin) + "</lastLogin>" +
			"</" + getType() + ">";
	}

	/**
	 * Returns an SQL INSERT string that allows the system to add
	 * the account to a relational database.
	 */
	public String asSql() {
		return	"INSERT INTO " + getType() + " (Username, " +
			"Password, FirstName, LastName, EmailAddress, " +
			"LastLogin) VALUES (\'" + normalizeSql(username) +
			"\', \'" + normalizeSql(password) +"\', \'" +
			normalizeSql(firstName) + "\', \'" +
			normalizeSql(lastName) + "\', \'" +
			normalizeSql(emailAddress) + "\', \'" +
			df.format(lastLogin) + "\');";
	}

	/**
	 * Returns an SQL UPDATE string that allows the system to update
	 * the account in a relational database.
	 */
	public String asSqlUpdate() {
		return  "UPDATE " + getType() + " SET Password = \'" +
			normalizeSql(password) + "\', FirstName = \'" +
			normalizeSql(firstName) + "\', LastName = \'" +
			normalizeSql(lastName) + "\', EmailAddress = \'" +
			normalizeSql(emailAddress) + "\', LastLogin = \'" +
			df.format(lastLogin) + "\' " + "WHERE Username = \'" +
			normalizeSql(username) + "\';";
	}
}
\end{lstlisting}

\subsubsection{Type refactoring}
% TODO

\newpage
