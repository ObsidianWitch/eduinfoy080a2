\subsection{JSONDatabase implementation}

First of all, the \emph{softarch.portal.db.json} package as well as the
\emph{JSONDatabase}, \emph{JSONDatabaseFactory}, \emph{RawJSONDatabase},
\emph{RegularJSONDatabase}, \emph{UserJSONDatabase} classes were
created. There is not much to say about the first 2 classes since they
are similar to the SQL ones. As for \emph{RawJSONDatabase} and
\emph{RegularJSONDatabase}, all their methods throw the following exception
since the assignment subject specified that we did not have to implement them.\\

\begin{lstlisting}
    throw new DatabaseException("not implemented");
\end{lstlisting}
\

Only the methods from the \emph{UserJSONDatabase} class were developed.
By doing so, we want to be able to create subscriptions
(\emph{FreeSubscription}, \emph{CheapSubscription},
\emph{ExpensiveSubscription}), login and logout.\\

The \emph{gson}\cite{cite:gson} library is used to parse and write JSON files.
The library's jar (\emph{gson-2.3.1.jar}) was added to the
\emph{WebContent/WEB-INF/lib} directory.

\subsubsection{First implementation}

\begin{framewarning}
The first implementation was done before the type refactoring.
\end{framewarning}

The constructor instantiate the \emph{usersFile} field which is an
instance of the \emph{File} class. This instance points to the
\emph{users.json} file where the data will be retrieved and stored. If
the file is not created, then a new empty JSON document is created.\\

\begin{lstlisting}[caption={empty users.json document}]
{"FreeSubscription":[],"CheapSubscription":[],"ExpensiveSubscription":[]}
\end{lstlisting}
\

The \emph{insert(UserProfile profile)} method first retrieve the content
of the JSON file and stores it in a \emph{JsonObject}. Then, we retrieve
the type of the \emph{UserProfile} in order to know in which array we
will insert the new profile. Next, the \emph{UserProfile} is transformed
in a \emph{JsonElement} instance with the
\emph{gson.toJsonTree(profile)} method and added to the appropriate
array. Finally, the modified JsonObject is written to the file.\\

\begin{lstlisting}[caption={FreeSubscription profile added}]
{
    "FreeSubscription":[
        {
            "username":"test",
            "password":"test",
            "firstName":"test",
            "lastName":"test",
            "emailAddress":"test",
            "lastLogin":"Mar 11, 2015 8:58:05 AM"
        }
    ],
    "CheapSubscription":[],
    "ExpensiveSubscription":[]
}
\end{lstlisting}
\

\begin{framehint}
\textbf{N.B.} We do not need an \emph{asJson()} method in \emph{UserProfile},
Gson uses reflection in order to be able to parse any class' instance.
\end{framehint}
\

The \emph{update(UserProfile profile)} method was implemented by first
calling the \emph{remove(UserProfile profile)} method, which removes a
profile based on the username, and then calling the
\emph{insert(UserProfile profile)} method.
Finally, \emph{userExists(String username)} just check whether
\emph{findUser(username)} returns null.\\

\begin{lstlisting}
public class UserJSONDatabase extends JSONDatabase implements UserDatabase {
    
    public File usersFile;

    public UserJSONDatabase(Properties properties) {
        super(properties);
        
        try {
            String dbUrl = properties.getProperty("dbJSONUrl");
            usersFile = new File(dbUrl + "/users.json");
            createJsonDocument();
        } catch(IOException e) {
            e.printStackTrace();
        }
    }
    
    private boolean createJsonDocument() throws IOException {
        boolean created = usersFile.createNewFile();
        if (!created) { return false; }
        
        JsonObject jo = new JsonObject();
        jo.add("FreeSubscription", new JsonArray());
        jo.add("CheapSubscription", new JsonArray());
        jo.add("ExpensiveSubscription", new JsonArray());
        
        Gson gson = new Gson();
        String jsonDocument = gson.toJson(jo);
        writeToFile(jsonDocument);
        
        return true;
    }

    public void insert(UserProfile profile) throws DatabaseException {
        Gson gson = new Gson();
        
        JsonObject jsonParsed = getUsersJson();
        
        String type = profile.getType();
        JsonElement jsonProfile = gson.toJsonTree(profile);
        
        JsonArray usersFromType = jsonParsed.getAsJsonArray(type);
        usersFromType.add(jsonProfile);

        try {
            writeToFile(jsonParsed.toString());
        } catch (IOException e) {
            throw new DatabaseException("IOException: " + e.getMessage());
        }
    }
    
    public void update(UserProfile profile) throws DatabaseException {
        remove(profile);
        insert(profile);
    }
    
    private void remove(UserProfile profile) throws DatabaseException {
        String type = profile.getType();
        JsonObject jsonParsed = getUsersJson();
        JsonArray usersFromType = jsonParsed.getAsJsonArray(type);
        
        for (JsonElement userProfileElement : usersFromType) {
            JsonObject userProfileObject = (JsonObject) userProfileElement;
            
            if (userProfileObject.get("username").getAsString()
                    .equals(profile.getUsername()))
            {
                usersFromType.remove(userProfileElement);
                break;
            }
        }
        
        try {
            writeToFile(jsonParsed.toString());
        } catch (IOException e) {
            throw new DatabaseException("IOException: " + e.getMessage());
        }
    }

    public UserProfile findUser(String username) throws DatabaseException {
        Gson gson = new Gson();
        
        try {
            JsonObject jsonParsed = getUsersJson();
            for (Map.Entry<String,JsonElement> entry : jsonParsed.entrySet()) {
                JsonArray types = (JsonArray) entry.getValue();
                
                for (JsonElement userProfileElement : types) {
                    JsonObject userProfileObject = (JsonObject) userProfileElement;
                    
                    if (userProfileObject.get("username").getAsString()
                            .equals(username))
                    {
                        return (UserProfile) gson.fromJson(
                            userProfileObject,
                            Class.forName("softarch.portal.data." + entry.getKey())
                        );
                    }
                }
            }
        } catch (JsonSyntaxException e) {
            e.printStackTrace();
        } catch (ClassNotFoundException e) {
            e.printStackTrace();
        }
        
        return null;
    }

    public boolean userExists(String username) throws DatabaseException {
        return findUser(username) != null;
    }
    
    private JsonObject getUsersJson() throws DatabaseException {
        JsonObject jsonParsed = null;
        
        try {
            JsonParser parser = new JsonParser();
            JsonElement jsonElement = parser.parse(new FileReader(usersFile));
            jsonParsed = jsonElement.getAsJsonObject();
        }
        catch (FileNotFoundException e) {
            throw new DatabaseException(
                "File not found Exception: " + e.getMessage()
            );
        }
        
        return jsonParsed;
    }
    
    private void writeToFile(String json) throws IOException {
        OutputStreamWriter out = new OutputStreamWriter(
                new FileOutputStream(usersFile)
        );
        
        out.write(json);
        out.close();
    }

}
\end{lstlisting}

\subsubsection{Second implementation}

With the type refactoring, we do not need to separate the different
subscriptions in multiple arrays. We just need to have the root element
be an array containing UserProfiles.

\begin{lstlisting}[caption={empty users.json document}]
[]
\end{lstlisting}
\

\begin{lstlisting}[caption={users.json example}]
[
    {
        "username":"op",
        "password":"op",
        "firstName":"op",
        "lastName":"op",
        "emailAddress":"op",
        "lastLogin":"Mar 11, 2015 9:53:40 PM",
        "type":"Operator"
    },
    {
        "username":"admin",
        "password":"admin",
        "firstName":"admin",
        "lastName":"admin",
        "emailAddress":"admin",
        "lastLogin":"Mar 11, 2015 9:53:48 PM",
        "type":"ExpertAdministrator"
    },
    {
        "username":"test",
        "password":"test",
        "firstName":"test",
        "lastName":"test",
        "emailAddress":"test",
        "lastLogin":"Mar 12, 2015 6:07:05 PM",
        "type":"FreeSubscription"
    }
]
\end{lstlisting}
