\section{Façade pattern}
The Façade pattern is useful to group objects which are part of a system, and
provide a unified interface to this system. The interface only exposes methods
that should be known to the outside. As such, a simplified interface to a
complex system is provided. The classes using the Façade, the layer of higher
level in our case, do not have to know about the internals of the complex
system, it just uses the simplified interface.\\

If we take \emph{DatabaseFacade} as an example, this class provides methods
to manipulate user profiles and data (check existence, get, insert, remove,
update). The facade do not give direct access to the database
(e.g. insert a row in a SQL table), and we do not even know which database is
used.\\

It facilitate switching between different implementations of a layer because the
pattern offers an unified interface. As such, if we want to switch between
layers, we need each layer to conform to this interface, and the classes using
this layer do not have to be modified.

\newpage
