\subsection{DatabaseFacade dependencies}

\begin{enumerate}
\item \emph{InternetFrontEnd.init()}: load properties (dbUser, dbPassword and
    dbUrl) from the configuration file (\emph{web\_portal.cfg}) and pass them to
    \emph{ApplicationFacade}
    
\item \emph{ApplicationFacade} pass the properties to \emph{DatabaseFacade}

\item \emph{DatabaseFacade} pass the properties to \emph{UserDatabase},
    \emph{RegularDatabase} and \emph{RawDatabase}
\end{enumerate}

The problem with the process described above is that properties which are SQL
related are retrieved at the presentation level and then explicitly given to
the logic level and finally to the database level. ``The database layer
interface should be generic, and not expose the specific requirements of the
layer'' (cf subject).\\

We would like to use other databases (e.g. JSON) than SQL ones, so we need to
remove or modify these dependencies from DatabaseFacade. The configuration need
to be retrieved from InternetFacade and given to the database layer by passing
trough the application layer, so, instead of passing the individual properties
as parameters, we can pass the \emph{Properties} object holding all properties.
It can then be passed to the 3 different databases (user, regular and raw).\\

The properties related to SQL in \emph{web\_portal.cfg} were also modified to
the following name pattern: dbSQL* (e.g. dbUser -> dbSQLUser).\\

\begin{lstlisting}
public class InternetFrontEnd extends HttpServlet {
	public void init() throws ServletException {
		...
		String cfgfile = getInitParameter("config_file");
		properties.load(new FileInputStream(cfgfile));
		ApplicationFacade appFacade = new ApplicationFacade(properties);
        ...
    }

public class ApplicationFacade {
    public ApplicationFacade(Properties properties) {
        DatabaseFacade dbFacade = new DatabaseFacade(properties);
        ...
    }
}

public DatabaseFacade(Properties properties) {
	public DatabaseFacade(Properties properties) {
		userDb    = new UserDatabase(properties);
		regularDb = new RegularDatabase(properties);
		rawDb     = new RawDatabase(properties);
	}
}

public class Database {
	public Database(Properties properties) {
		this.dbUser	= properties.getProperty("dbSQLUser");
		this.dbPassword	= properties.getProperty("dbSQLPassword");
		this.dbUrl	= properties.getProperty("dbSQLUrl");
	}
}
\end{lstlisting}

\newpage
